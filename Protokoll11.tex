% Für Bindekorrektur als optionales Argument "BCORfaktormitmaßeinheit", dann
% sieht auch Option "twoside" vernünftig aus
% Näheres zu "scrartcl" bzw. "scrreprt" und "scrbook" siehe KOMA-Skript Doku
\documentclass[12pt,a4paper,titlepage,headinclude,bibtotoc]{scrartcl}


%---- Allgemeine Layout Einstellungen ------------------------------------------

% Für Kopf und Fußzeilen, siehe auch KOMA-Skript Doku
\usepackage[komastyle]{scrpage2}
\pagestyle{scrheadings}
\automark[section]{chapter}
\setheadsepline{0.5pt}[\color{black}]

%keine Einrückung
\parindent0pt

%Einstellungen für Figuren- und Tabellenbeschriftungen
\setkomafont{captionlabel}{\sffamily\bfseries}
\setcapindent{0em}

\usepackage{caption}

%---- Weitere Pakete -----------------------------------------------------------
% Die Pakete sind alle in der TeX Live Distribution enthalten. Wichtige Adressen
% www.ctan.org, www.dante.de

% Sprachunterstützung
\usepackage[ngerman]{babel}

% Benutzung von Umlauten direkt im Text
% entweder "latin1" oder "utf8"
\usepackage[utf8]{inputenc}

% Pakete mit Mathesymbolen und zur Beseitigung von Schwächen der Mathe-Umgebung
\usepackage{latexsym,exscale,amssymb,amsmath}

% Weitere Symbole
\usepackage[nointegrals]{wasysym}
\usepackage{eurosym}

% Anderes Literaturverzeichnisformat
%\usepackage[square,sort&compress]{natbib}

% Für Farbe
\usepackage{color}

% Zur Graphikausgabe
%Beipiel: \includegraphics[width=\textwidth]{grafik.png}
\usepackage{graphicx}

% Text umfließt Graphiken und Tabellen
% Beispiel:
% \begin{wrapfigure}[Zeilenanzahl]{"l" oder "r"}{breite}
%   \centering
%   \includegraphics[width=...]{grafik}
%   \caption{Beschriftung} 
%   \label{fig:grafik}
% \end{wrapfigure}
\usepackage{wrapfig}

% Mehrere Abbildungen nebeneinander
% Beispiel:
% \begin{figure}[htb]
%   \centering
%   \subfigure[Beschriftung 1\label{fig:label1}]
%   {\includegraphics[width=0.49\textwidth]{grafik1}}
%   \hfill
%   \subfigure[Beschriftung 2\label{fig:label2}]
%   {\includegraphics[width=0.49\textwidth]{grafik2}}
%   \caption{Beschriftung allgemein}
%   \label{fig:label-gesamt}
% \end{figure}
\usepackage{subfigure}
\usepackage{adjustbox}

% Caption neben Abbildung
% Beispiel:
% \sidecaptionvpos{figure}{"c" oder "t" oder "b"}
% \begin{SCfigure}[rel. Breite (normalerweise = 1)][hbt]
%   \centering
%   \includegraphics[width=0.5\textwidth]{grafik.png}
%   \caption{Beschreibung}
%   \label{fig:}
% \end{SCfigure}
\usepackage{sidecap}

% Befehl für "Entspricht"-Zeichen
\newcommand{\corresponds}{\ensuremath{\mathrel{\widehat{=}}}}

%Für chemische Formeln (von www.dante.de)
%% Anpassung an LaTeX(2e) von Bernd Raichle
\makeatletter
\DeclareRobustCommand{\chemical}[1]{%
  {\(\m@th
   \edef\resetfontdimens{\noexpand\)%
       \fontdimen16\textfont2=\the\fontdimen16\textfont2
       \fontdimen17\textfont2=\the\fontdimen17\textfont2\relax}%
   \fontdimen16\textfont2=2.7pt \fontdimen17\textfont2=2.7pt
   \mathrm{#1}%
   \resetfontdimens}}
\makeatother

%Si Einheiten
\usepackage{siunitx}

%c++ Code einbinden
\usepackage{listings}
\lstset{numbers=left, numberstyle=\tiny, numbersep=5pt}

%Differential
\newcommand{\dif}{\ensuremath{\mathrm{d}}}

%Boxen,etc.
\usepackage{fancybox}
\usepackage{empheq}

%Fußnoten auf gleiche Seite
\interfootnotelinepenalty=1000

%Dateien aus Unterverzeichnissen
\usepackage{import}

%Bibliography \bibliography{literatur} und \cite{gerthsen}
%\usepackage{cite}
\usepackage{babelbib}
\selectbiblanguage{ngerman}

\begin{document}

\begin{titlepage}
\centering
\textsc{\Large Anfängerpraktikum der Fakultät für
  Physik,\\[1.5ex] Universität Göttingen}

\vspace*{4.2cm}

\rule{\textwidth}{1pt}\\[0.5cm]
{\huge \bfseries
  Messung von großen Widerständen\\[1.5ex]
  Protokoll:}\\[0.5cm]
\rule{\textwidth}{1pt}

\vspace*{3.0cm}

\begin{Large}
\begin{tabular}{ll}
Praktikant:
 	&  Felix Kurtz\\
 	&  Michael Lohmann\\

E-Mail: 
	&  felix.kurtz@stud.uni-goettingen.de\\
	& m.lohmann@stud.uni-goettingen.de\\

 Betreuer: & Björn Klaas\\
 Versuchsdatum: &  03.09.2014\\
\end{tabular}
\end{Large}

\vspace*{0.8cm}

\begin{Large}
\fbox{
  \begin{minipage}[t][2.5cm][t]{6cm} 
    Testat:
  \end{minipage}
}
\end{Large}

\end{titlepage}

\tableofcontents

\newpage

\section{Einleitung}
\label{sec:einleitung}
Um einen Widerstand zu messen, nutzt man meistens das Ohmsche Gesetz.
Ist der Widerstand jedoch hochohmig, stößt dieses Verfahren an seine Grenzen.
Man arbeitet mit hohen Spannungen und kleinen Strömen.
Außerdem sind die Innenwiderstände der Messgeräte ein großer Störfaktor.
Deshalb werden wir in diesem Versuch lernen, wie man das besser machen kann. 

\section{Theorie}
\label{sec:theorie}
\subsection{Messung mittels einem Kondensator}

\begin{align}
	Q(t)=Q_0 \exp \left(-\frac{t}{RC}\right)
\end{align}
Kennt man die Kapazität $C$ und die Ladung, die sich auf dem Kondensator befindet, zu zwei verschiedenen Zeitpunkten $t_1$ und $t_2$, kann man also den Widerstand $R$ berechnen, über den der Strom abfließt:
\begin{align}
	R=-\frac{t_2-t_1}{C\cdot\ln\frac{Q(t_2)}{Q(t_1)}}
\end{align}
Der hier verwendete Kondenstor hat einen Plattenradius $r=0.1~\si{\meter}$, einen Plattenabstand $d=0.005~\si{\meter}$ und eine Plattenzahl $n=65$.
Für die Berechnung der Kapazität müssen also Randeffekte betrachtet werden.
Dabei wird diese Formel verwendet:
\begin{align}	
	C_n=(n-1)\varepsilon_0\varepsilon_r\left[\frac{\pi r^2}{d}+r\left(\ln\frac{16\pi r}{d}-1\right)\right]
\end{align}
%ergibt sich eine Kapazität $C=3.896~\si{\nano\farad}$.

\subsection{Analoger Stromintegrator}
Nach der \textit{Kirchhoffschen Knotenregel} bei S gilt $I_R+I_C=0$.
Mit den folgenden Beziehungen der Ströme $I_R=U_E/R$ und $I_C=\dot{Q}_C=C\dot{U}_A$ erhält man:
\begin{align}
	U_A=-\frac{1}{RC}\int \limits_{t_0}^t U_E \,\dif t
\end{align}

\subsection{RLC-Schwingkreis}
\begin{align}
	\ddot{Q}+2\beta\dot{Q}+\omega_0^2 Q=0
\end{align}
\begin{align*}
	\beta=\frac{R_L}{2L} \quad , \quad 
	\omega_0=\sqrt{\frac{1}{LC}} \quad , \quad
	\omega=\sqrt{\omega_0^2-\beta^2}
\end{align*}
Mit dem Logarithmischen Dekrement $\Lambda=\beta T$ ergibt sich für die Induktivität der Spule
\begin{align}
	L=\frac{1}{C\omega_0^2}=\frac{1}{C(\omega^2+\beta^2)}=\frac{1}{C}\frac{T^2}{4\pi^2+\Lambda^2}
\end{align}
\begin{align}
	L=\mu_0 \cdot A \cdot \left(\frac{n}{l}\right)^2
\end{align}

\section{Durchführung}
\label{sec:durchfuehrung}

\section{Auswertung}
\label{sec:auswertung}
\subsection{Kalibration des Ladungsmessgerätes}
\begin{figure}
 \centering
 \input{Kalibration.tex}
 \caption{Skalenteile des Messgeräts in Abhängigkeit der geflossenen Ladung}
 \label{fig:Kalibration}
\end{figure}

$$m=0.8729 \pm 0.0017 ~\text{Skt.}/\si{\micro\coulomb}$$

\subsection{Berechnung von $\varepsilon_0$}

\begin{empheq}[box=\shadowbox*]{align}
	\varepsilon_0= \left(9.19 \pm 0.07\right)\cdot 10^{-12}\,\si[per-mode=fraction]{\ampere\second\per\volt\per\meter}
\end{empheq}

\begin{align*}
	C&=\frac{Q}{U}\\
	C&=\left(4.04 \pm 0.03\right) \cdot 10^{-9}\\
	\sigma_{C}&=\frac{1}{U^{2}} \cdot \sqrt{Q^{2} \cdot \sigma_{U}^{2} + \sigma_{Q}^{2} \cdot U^{2}}
\end{align*}


\subsection{Entladung des Kondensators}
\begin{figure}[!htb]
	\centering
	\input{Entladen1.tex}
	\caption{Entladung des Kondesators über $R_x$ und $R_\text{iso}$}
\end{figure}

\begin{figure}[!htb]
	\centering
	\input{Entladen2.tex}
	\caption{Entladung des Kondesators über den Isolationswiderstand $R_\text{iso}$}
\end{figure}

\begin{align*}
	R_\text{iso}&=- \frac{1}{C \cdot m_\text{iso}}\\
	R_\text{iso}&=\left(1.5 \pm 0.1\right) \times 10^{10}\\
	\sigma_{R_\text{iso}}&=\frac{1}{C^{2} \cdot m_\text{iso}^{2}} \cdot \sqrt{C^{2} \cdot \sigma_{m_\text{iso}}^{2} + m_\text{iso}^{2} \cdot \sigma_{C}^{2}}
\end{align*}

\begin{align*}
	R&=- \frac{1}{C \cdot m}\\
	R&=\left(2.4 \pm 0.2\right) \times 10^{9}\\
	\sigma_{R}&=\frac{1}{C^{2} \cdot m^{2}} \cdot \sqrt{C^{2} \cdot \sigma_{m}^{2} + m^{2} \cdot \sigma_{C}^{2}}
\end{align*}

\begin{align*}
	R_x&=\frac{1}{\frac{1}{R} - \frac{1}{R_\text{iso}}}\\
	R_x&=\left(2.8 \pm 0.4\right) \times 10^{9}\\
	\sigma_{R_x}&=\frac{1}{\left(R_\text{iso} - R\right)^{2}} \cdot \sqrt{R_\text{iso}^{4} \cdot \sigma_{R}^{2} + R^{4} \cdot \sigma_{R_\text{iso}}^{2}}
\end{align*}

\subsection{Schwingkreise}
\subsubsection{Schritt 5}
\begin{align*}
	R_\text{oszi}&=R_2 \cdot \left(\frac{m_\text{ges}}{m_\text{oszi}} - 1\right)\\
	R_\text{oszi}&=\left(9.0 \pm 0.6\right) \times 10^{5}\\
	\sigma_{R_\text{oszi}}&=\frac{1}{m_\text{oszi}^{2}} \cdot \sqrt{m_\text{ges}^{2} \cdot R_2^{2} \cdot \sigma_{m_\text{oszi}}^{2} + m_\text{oszi}^{2} \cdot \left(R_2^{2} \cdot \sigma_{m_\text{ges}}^{2} + \sigma_{R_2}^{2} \cdot \left(m_\text{ges} - m_\text{oszi}\right)^{2}\right)}
\end{align*}

\begin{align*}
	C&=- \frac{1}{m_\text{oszi} \cdot R_\text{oszi}}\\
	C&=\left(4.3 \pm 0.3\right) \times 10^{-9}\\
	\sigma_{C}&=\frac{1}{m_\text{oszi}^{2} \cdot R_\text{oszi}^{2}} \cdot \sqrt{m_\text{oszi}^{2} \cdot \sigma_{R_\text{oszi}}^{2} + R_\text{oszi}^{2} \cdot \sigma_{m_\text{oszi}}^{2}}
\end{align*}

\subsubsection{Schritt 6}
\begin{align*}
	R_x&=\frac{R_\text{oszi}}{\frac{m_x}{m_\text{oszi}} - 1}\\
	R_x&=\left(-0.3 \pm 1.2\right) \times 10^{9}\\
	\sigma_{R_x}&=\frac{1}{\left(m_x - m_\text{oszi}\right)^{2}} \cdot \sqrt{m_\text{oszi}^{2} \cdot \sigma_{R_\text{oszi}}^{2} \cdot \left(m_x - m_\text{oszi}\right)^{2} + R_\text{oszi}^{2} \cdot \left(m_x^{2} \cdot \sigma_{m_\text{oszi}}^{2} + m_\text{oszi}^{2} \cdot \sigma_{m_x}^{2}\right)}
\end{align*}

\subsubsection{Schritt 7}
\begin{align*}
L&=\frac{T^{2}}{C \cdot \left(\beta^{2} \cdot T^{2} + 4 \cdot \pi^{2}\right)}\\
\sigma_{L}&=\frac{T}{C^{2} \cdot \left(\beta^{2} \cdot T^{2} + 4 \cdot \pi^{2}\right)^{2}} \cdot \sqrt{4 \cdot \beta^{2} \cdot C^{2} \cdot \sigma_{\beta}^{2} \cdot T^{6} + 64 \cdot \pi^{4} \cdot C^{2} \cdot \sigma_{T}^{2} + \sigma_{C}^{2} \cdot T^{2} \cdot \left(\beta^{2} \cdot T^{2} + 4 \cdot \pi^{2}\right)^{2}}
\end{align*}
	
\begin{align*}
	L&=9.6 \pm 0.7\\
	L&=0.018 \pm 0.001
\end{align*}

\begin{align*}
R_L&=2 \cdot \beta \cdot L\\
\sigma_{R_L}&=2 \cdot \sqrt{\beta^{2} \cdot \sigma_{L}^{2} + L^{2} \cdot \sigma_{\beta}^{2}}
\end{align*}

\begin{align*}
	R_L=\left(6.1 \pm 0.5\right) \times 10^{3}\\
	R_L=\left(1.6 \pm 0.1\right) \times 10^{2}
\end{align*}

\subsubsection{Schritt 9}
\begin{align*}
C_2&=\frac{C_\text{Pl.}}{m_g} \cdot m_c\\
C_2&=\left(3.1 \pm 0.2\right) \times 10^{-9}\\
\sigma_{C_2}&=\frac{1}{m_g^{2}} \cdot \sqrt{C_\text{Pl.}^{2} \cdot m_c^{2} \cdot \sigma_{m_g}^{2} + m_g^{2} \cdot \left(C_\text{Pl.}^{2} \cdot \sigma_{m_c}^{2} + m_c^{2} \cdot \sigma_{C_\text{Pl.}}^{2}\right)}
\end{align*}

\section{Diskussion}
\label{sec:diskussion}

\section{Anhang}
\begin{figure}[!htb]
	\centering
	\input{Drosselspule.tex}
	\caption{\textit{Drosselspule}: Extrema des Spannungsverlauf logarithmisch gegen die Zeit}
\end{figure}

\begin{figure}[!htb]
	\centering
	\input{Luftspule.tex}
	\caption{\textit{Luftspule}: Extrema des Spannungsverlauf logarithmisch gegen die Zeit}
\end{figure}

\end{document}
